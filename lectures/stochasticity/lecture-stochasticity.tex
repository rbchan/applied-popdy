\documentclass[color=usenames,dvipsnames]{beamer}\usepackage[]{graphicx}\usepackage[]{xcolor}
% maxwidth is the original width if it is less than linewidth
% otherwise use linewidth (to make sure the graphics do not exceed the margin)
\makeatletter
\def\maxwidth{ %
  \ifdim\Gin@nat@width>\linewidth
    \linewidth
  \else
    \Gin@nat@width
  \fi
}
\makeatother

\definecolor{fgcolor}{rgb}{0.345, 0.345, 0.345}
\newcommand{\hlnum}[1]{\textcolor[rgb]{0.686,0.059,0.569}{#1}}%
\newcommand{\hlsng}[1]{\textcolor[rgb]{0.192,0.494,0.8}{#1}}%
\newcommand{\hlcom}[1]{\textcolor[rgb]{0.678,0.584,0.686}{\textit{#1}}}%
\newcommand{\hlopt}[1]{\textcolor[rgb]{0,0,0}{#1}}%
\newcommand{\hldef}[1]{\textcolor[rgb]{0.345,0.345,0.345}{#1}}%
\newcommand{\hlkwa}[1]{\textcolor[rgb]{0.161,0.373,0.58}{\textbf{#1}}}%
\newcommand{\hlkwb}[1]{\textcolor[rgb]{0.69,0.353,0.396}{#1}}%
\newcommand{\hlkwc}[1]{\textcolor[rgb]{0.333,0.667,0.333}{#1}}%
\newcommand{\hlkwd}[1]{\textcolor[rgb]{0.737,0.353,0.396}{\textbf{#1}}}%
\let\hlipl\hlkwb

\usepackage{framed}
\makeatletter
\newenvironment{kframe}{%
 \def\at@end@of@kframe{}%
 \ifinner\ifhmode%
  \def\at@end@of@kframe{\end{minipage}}%
  \begin{minipage}{\columnwidth}%
 \fi\fi%
 \def\FrameCommand##1{\hskip\@totalleftmargin \hskip-\fboxsep
 \colorbox{shadecolor}{##1}\hskip-\fboxsep
     % There is no \\@totalrightmargin, so:
     \hskip-\linewidth \hskip-\@totalleftmargin \hskip\columnwidth}%
 \MakeFramed {\advance\hsize-\width
   \@totalleftmargin\z@ \linewidth\hsize
   \@setminipage}}%
 {\par\unskip\endMakeFramed%
 \at@end@of@kframe}
\makeatother

\definecolor{shadecolor}{rgb}{.97, .97, .97}
\definecolor{messagecolor}{rgb}{0, 0, 0}
\definecolor{warningcolor}{rgb}{1, 0, 1}
\definecolor{errorcolor}{rgb}{1, 0, 0}
\newenvironment{knitrout}{}{} % an empty environment to be redefined in TeX

\usepackage{alltt}
%\documentclass[color=usenames,dvipsnames,handout]{beamer}

%\usepackage[roman]{../pres1}
\usepackage[sans]{../pres1}
\usepackage{graphicx}
\usepackage{bm}
%\usepackage{changepage}

%\newcommand{\wide}{\column{\dimexpr\paperwidth}} % Must be in columns environment

\mode<handout>{
  \usetheme{default}
%  \setbeamercolor{background canvas}{bg=black!5}
%  \pgfpagesuselayout{4 on 1}[letterpaper,landscape,border shrink=2.5mm]
%  \pgfpagesuselayout{2 on 1}[letterpaper,border shrink=10mm]
}




\IfFileExists{upquote.sty}{\usepackage{upquote}}{}
\begin{document}


\begin{frame}[plain]
  \begin{center}
    {\huge Stochasticity} \\
    \vfill
    \includegraphics[height=3.8cm,keepaspectratio]{figs/pelagic-cormorants} \hspace{0.1cm}
    \includegraphics[height=3.8cm,keepaspectratio]{figs/kittiwakes} %\hspace{0.5cm}
  \end{center}
\end{frame}




\section{Introduction}


\begin{frame}[plain]
  \frametitle{Learning objectives}
  \tableofcontents%[currentsection]
\end{frame}




\begin{frame}
  \frametitle{Random variables}
  \large
  A random variable is a variable whose value can't be predicted
  with certainty. \\
  \pause
  \vfill
  Examples
  \begin{itemize}
    \item Weather
    \item Our own behavior
    \item Population size
  \end{itemize}
  \pause
  \vfill
  Is the universe random? Or is just too complex to be predicted with certainty? 
\end{frame}




\begin{frame}
  \frametitle{Probability distributions}
  \large
  {A random variable ($X$) can be described by a probability distribution. \\}
%  \pause
  \vspace{1cm}
  {There are many types of probability distributions}
  \begin{itemize}
    \item Normal (or Gaussian)
    \item Poisson
    \item Binomial
    \item Multinomial
    \item etc\dots
  \end{itemize}
\end{frame}





\begin{frame}[fragile]
  \frametitle{Normal (Gaussian) Distribution}
\[
  X \sim \mbox{Normal}(\mu=0, \sigma^2=1)
\]

\vspace{-1cm}
%\centering
\begin{center}
  \includegraphics<1 | handout:0>[width=\textwidth]{figs/norm/norm0} %\\
  \includegraphics<2 | handout:0>[width=\textwidth]{figs/norm/norm1} %\\
  \includegraphics<3 | handout:0>[width=\textwidth]{figs/norm/norm2} %\\
  \includegraphics<4 | handout:0>[width=\textwidth]{figs/norm/norm3} %\\
  \includegraphics<5 | handout:0>[width=\textwidth]{figs/norm/norm4} %\\
  \includegraphics<6 | handout:0>[width=\textwidth]{figs/norm/norm5} %\\
  \includegraphics<7 | handout:0>[width=\textwidth]{figs/norm/norm6} %\\
  \includegraphics<8 | handout:0>[width=\textwidth]{figs/norm/norm7} %\\
  \includegraphics<9 | handout:0>[width=\textwidth]{figs/norm/norm8} %\\
  \includegraphics<10 | handout:0>[width=\textwidth]{figs/norm/norm9} %\\
  \includegraphics<11->[width=\textwidth]{figs/norm/norm10} %\\
\end{center}

\end{frame}






\begin{frame}[fragile]
  \frametitle{Normal (Gaussian) distribution}
  \vspace{-0.7cm}
  \begin{center}
\begin{knitrout}
\definecolor{shadecolor}{rgb}{0.969, 0.969, 0.969}\color{fgcolor}
\includegraphics[width=\maxwidth]{figure/normals-1} 
\end{knitrout}
  \end{center}
%\includegraphics[width=\textwidth]{figure/norm1-1}
\end{frame}





\begin{frame}[fragile]
  \frametitle{A purely stochastic model}
  \vspace{-3mm}
  \[
    N_t \sim \mbox{Normal}(\mu=50, \sigma^2=1)
  \]
  \vspace{-8mm}

\begin{center}
%  \includegraphics[width=0.7\textwidth]{stochasticity-normpop}
  \includegraphics[width=0.9\textwidth]{figure/normpop-1}
\end{center}
\note{Amazing that this looks more realistic than a theoretical model}
\end{frame}



\begin{frame}
  \frametitle{Two important types of stochasticity}
  {\bf Environmental stochasticity}
  \begin{itemize}
    \item Random variation in demographic rates (such as birth
      and death rates)
    \item Caused by unpredictable changes in weather, habitat,
      etc\dots conditions among years
    \item[]
  \end{itemize}
  \pause
  {\bf Demographic stochasticity}
  \begin{itemize}
    \item Random variation in demographic outcomes (such as the number
      of births and deaths)
    \item Occurs even when demographic rates are constant 
    \item[]
  \end{itemize}
\end{frame}







% \section{Geometric Growth}
\section{Environmental stochasticity}





% \begin{frame}[fragile]
%   \frametitle{\large Geometric growth with environmental stochasticity}
%   \Large
% \[
%   N_{t+1} = N_t + N_tr + X_t
% \] \\
% %\vspace{12pt}
% %{\large \centering where \\}
% %\vspace{-12pt}
% \[
%   X_t \sim \mbox{Normal}(0, \sigma_e^2)
% \]
% \pause
% %\vfill
% % {\tt \R} code:
% R code
% <<geo-env,eval=FALSE,size='small'>>=
% nYears <- 20
% N <- X <- rep(NA, nYears)  ## Create empty N and X
% N[1] <- 100                ## Initial value of N
% r <- 0.1                   ## Growth rate
% sigma.e <- 10              ## StdDev of enviro variation
% for(t in 2:nYears) {
%     X[t-1] <- rnorm(n=1, mean=0, sd=sigma.e)
%     N[t] <- N[t-1] + N[t-1]*r + X[t-1]
% }
% @
% \end{frame}






% \begin{frame}[fragile]
%   \frametitle{Example $N_0=100$, $r=0.1$, $\mu=0$, $\sigma_e^2=100$}
% <<geo-e,echo=FALSE,include=FALSE,cache=TRUE>>=
% r <- 0.1
% sigma.e <- 10
% T <- 20
% if(!dir.exists("figs/exp-e"))
%     dir.create("figs/exp-e")
% set.seed(350)
% for(i in 1:10) {
%     N <- integer(T+1)
%     N[1] <- 100
%     fl <- paste("figs/exp-e/exp-e", i, ".pdf", sep="")
%     for(t in 1:T) {
%         X.t <- rnorm(1, 0, sigma.e)
%         N[t+1] <- N[t] + N[t]*r + X.t
%     }
%     pdf(fl, width=8, heigh=6)
%     par(mai=c(0.9,0.9,0.1,0.1))
%     plot(0:T, N, type="b", xlab="Time", ylab="Population size (N)",
%          ylim=c(0, 1000), cex.lab=1.4)
%     plot(function(x) N[1]*(1+r)^x, 0, T+1, add=TRUE)
%     abline(h=0, col=gray(0.8))
%     dev.off()
% }
% @
% \vspace{-0.1cm}
% \begin{center}
%   \includegraphics<1 | handout:0>[width=\textwidth]{figs/exp-e/exp-e1}
%   \includegraphics<2 | handout:0>[width=\textwidth]{figs/exp-e/exp-e2}
%   \includegraphics<3 | handout:0>[width=\textwidth]{figs/exp-e/exp-e3}
%   \includegraphics<4 | handout:0>[width=\textwidth]{figs/exp-e/exp-e4}
%   \includegraphics<5 | handout:0>[width=\textwidth]{figs/exp-e/exp-e5}
%   \includegraphics<6 | handout:0>[width=\textwidth]{figs/exp-e/exp-e6}
%   \includegraphics<7 | handout:0>[width=\textwidth]{figs/exp-e/exp-e7}
%   \includegraphics<8 | handout:0>[width=\textwidth]{figs/exp-e/exp-e8}
%   \includegraphics<9 | handout:0>[width=\textwidth]{figs/exp-e/exp-e9}
%   \includegraphics<10>[width=\textwidth]{figs/exp-e/exp-e10}
% \end{center}
% \end{frame}





% \begin{frame}[fragile]
%   \frametitle{Example $N_0=100$, $r=0.1$, $\mu=0$, $\sigma_e^2=10000$}
% <<geo-e2,fig.show='hide',echo=FALSE,cache=TRUE>>=
% r <- 0.1
% sigma.e <- 100
% T <- 20
% if(!dir.exists("figs/exp-e2"))
%     dir.create("figs/exp-e2")
% set.seed(540)
% for(i in 1:10) {
%     N <- integer(T+1)
%     N[1] <- 100
%     fl <- paste("figs/exp-e2/exp-e", i, ".pdf", sep="")
%     for(t in 1:T) {
%         X.t <- rnorm(1, 0, sigma.e)
%         N[t+1] <- max(N[t] + N[t]*r + X.t, 0)
%     }
%     pdf(fl, width=8, height=6)
%     par(mai=c(0.9,0.9,0.1,0.1))
%     plot(0:T, N, type="b", xlab="Time", ylab="Population size (N)",
%          ylim=c(0, 1000), cex.lab=1.4)
%     plot(function(x) N[1]*(1+r)^x, 0, T+1, add=TRUE)
%     abline(h=0, col=gray(0.8))
%     dev.off()
% }
% @
% \vspace{-0.1cm}
% \begin{center}
%   \includegraphics<1 | handout:0>[width=\textwidth]{figs/exp-e2/exp-e1}
%   \includegraphics<2 | handout:0>[width=\textwidth]{figs/exp-e2/exp-e2}
%   \includegraphics<3 | handout:0>[width=\textwidth]{figs/exp-e2/exp-e3}
%   \includegraphics<4 | handout:0>[width=\textwidth]{figs/exp-e2/exp-e4}
%   \includegraphics<5 | handout:0>[width=\textwidth]{figs/exp-e2/exp-e5}
%   \includegraphics<6 | handout:0>[width=\textwidth]{figs/exp-e2/exp-e6}
%   \includegraphics<7 | handout:0>[width=\textwidth]{figs/exp-e2/exp-e7}
%   \includegraphics<8 | handout:0>[width=\textwidth]{figs/exp-e2/exp-e8}
%   \includegraphics<9 | handout:0>[width=\textwidth]{figs/exp-e2/exp-e9}
%   \includegraphics<10>[width=\textwidth]{figs/exp-e2/exp-e10}
% \end{center}
% \end{frame}







\begin{frame}[fragile]
%  \frametitle{\large Geometric growth with demographic stochasticity}
  \frametitle{\large Geometric growth with environmental stochasticity}
  \Large
\[
  N_{t+1} = N_t + N_t r_t
\]
%\vspace{0.3cm} \pause
%{\large \centering where \\}
%\vspace{-12pt}
\[
  r_t \sim \mbox{Normal}(\bar{r}, \sigma^2)
\] 
%\normalsize
%{\tt \R} code:
R code
\begin{knitrout}\small
\definecolor{shadecolor}{rgb}{0.969, 0.969, 0.969}\color{fgcolor}\begin{kframe}
\begin{alltt}
\hldef{nYears} \hlkwb{<-} \hlnum{20}
\hldef{N} \hlkwb{<-} \hldef{r} \hlkwb{<-} \hlkwd{rep}\hldef{(}\hlnum{NA}\hldef{, nYears)}  \hlcom{## Create empty N and r}
\hldef{N[}\hlnum{1}\hldef{]} \hlkwb{<-} \hlnum{100}                \hlcom{## Initial value of N}
\hldef{r.bar} \hlkwb{<-} \hlnum{0.5}               \hlcom{## Average growth rate}
\hldef{sigma} \hlkwb{<-} \hlnum{0.1}               \hlcom{## StdDev of growth rate}
\hlkwa{for}\hldef{(t} \hlkwa{in} \hlnum{2}\hlopt{:}\hldef{nYears) \{}
    \hldef{r[t}\hlopt{-}\hlnum{1}\hldef{]} \hlkwb{<-} \hlkwd{rnorm}\hldef{(}\hlkwc{n}\hldef{=}\hlnum{1}\hldef{,} \hlkwc{mean}\hldef{=r.bar,} \hlkwc{sd}\hldef{=sigma)}
    \hldef{N[t]} \hlkwb{<-} \hldef{N[t}\hlopt{-}\hlnum{1}\hldef{]} \hlopt{+} \hldef{N[t}\hlopt{-}\hlnum{1}\hldef{]}\hlopt{*}\hldef{r[t}\hlopt{-}\hlnum{1}\hldef{]}
\hldef{\}}
\end{alltt}
\end{kframe}
\end{knitrout}
\end{frame}







\begin{frame}[fragile]
  \frametitle{Example $N_0=100$, $\bar{r}=0.5$, $\sigma^2=0.01$}

\vspace{-0.3cm}
\begin{center}
  \includegraphics<1 | handout:0>[width=\textwidth]{figs/exp-d/exp-d1}
  \includegraphics<2 | handout:0>[width=\textwidth]{figs/exp-d/exp-d2}
  \includegraphics<3 | handout:0>[width=\textwidth]{figs/exp-d/exp-d3}
  \includegraphics<4 | handout:0>[width=\textwidth]{figs/exp-d/exp-d4}
  \includegraphics<5 | handout:0>[width=\textwidth]{figs/exp-d/exp-d5}
  \includegraphics<6 | handout:0>[width=\textwidth]{figs/exp-d/exp-d6}
  \includegraphics<7 | handout:0>[width=\textwidth]{figs/exp-d/exp-d7}
  \includegraphics<8 | handout:0>[width=\textwidth]{figs/exp-d/exp-d8}
  \includegraphics<9 | handout:0>[width=\textwidth]{figs/exp-d/exp-d9}
  \includegraphics<10>[width=\textwidth]{figs/exp-d/exp-d10}
\end{center}
\end{frame}






\begin{frame}[fragile]
  \frametitle{Example $N_0=100$, $\bar{r}=0.5$, $\sigma^2=0.25$}

\vspace{-0.3cm}
\begin{center}
  \includegraphics<1 | handout:0>[width=\textwidth]{figs/exp-d2/exp-d1}
  \includegraphics<2 | handout:0>[width=\textwidth]{figs/exp-d2/exp-d2}
  \includegraphics<3 | handout:0>[width=\textwidth]{figs/exp-d2/exp-d3}
  \includegraphics<4 | handout:0>[width=\textwidth]{figs/exp-d2/exp-d4}
  \includegraphics<5 | handout:0>[width=\textwidth]{figs/exp-d2/exp-d5}
  \includegraphics<6 | handout:0>[width=\textwidth]{figs/exp-d2/exp-d6}
  \includegraphics<7 | handout:0>[width=\textwidth]{figs/exp-d2/exp-d7}
  \includegraphics<8 | handout:0>[width=\textwidth]{figs/exp-d2/exp-d8}
  \includegraphics<9 | handout:0>[width=\textwidth]{figs/exp-d2/exp-d9}
  \includegraphics<10>[width=\textwidth]{figs/exp-d2/exp-d10}
\end{center}
\end{frame}









\begin{frame}
  \frametitle{\large Logistic growth with environmental stochasticity}
  \Large
  %\centering
  Imagine the carrying capacity fluctuates randomly from year to year. \\
\[
  N_{t+1} = N_t + N_tr_{max}(1 - N_t/K_t)
\]
%{\large \centering where \\}
%\vspace{-12pt}
\[
  K_t \sim \mbox{Normal}(\bar{K}, \sigma^2)
\]
\end{frame}






\begin{frame}[fragile]
  \frametitle{Logistic example, $r_{max}=0.2$, $\bar{K}=200$, $\sigma^2=50$}

\vspace{-0.2cm}
\begin{center}
  \includegraphics<1 | handout:0>[width=\textwidth]{figs/lg-d/lg-d1}
  \includegraphics<2 | handout:0>[width=\textwidth]{figs/lg-d/lg-d2}
  \includegraphics<3 | handout:0>[width=\textwidth]{figs/lg-d/lg-d3}
  \includegraphics<4 | handout:0>[width=\textwidth]{figs/lg-d/lg-d4}
  \includegraphics<5 | handout:0>[width=\textwidth]{figs/lg-d/lg-d5}
  \includegraphics<6 | handout:0>[width=\textwidth]{figs/lg-d/lg-d6}
  \includegraphics<7 | handout:0>[width=\textwidth]{figs/lg-d/lg-d7}
  \includegraphics<8 | handout:0>[width=\textwidth]{figs/lg-d/lg-d8}
  \includegraphics<9 | handout:0>[width=\textwidth]{figs/lg-d/lg-d9}
  \includegraphics<10>[width=\textwidth]{figs/lg-d/lg-d10}
\end{center}
\end{frame}




\section{Demographic stochasticity}



\begin{frame}[fragile]
  \frametitle{Demographic stochasticity}
  We can make the BIDE model stochastic by modeling random variation
  in the number of individuals that are born and die each year (we'll
  ignore movement):
  \[
     N_{t+1} = N_t + B_t - D_t
  \]
  \pause
  \vfill
  What distributions should we use for $B_t$ and $D_t$? 
  \pause
  \[
    B_t \sim \mathrm{Poisson}(N_t\times b)
  \]
  \[
    D_t \sim \mathrm{Binomial}(N_t, d)
  \]
\end{frame}


\begin{frame}
  \frametitle{Poisson distribution}
  The Poisson distribution is useful for data that are non-negative
  integers. \\
  \vfill
  It has a single parameter that describes the expected value of the
  random outcomes. \\
  \vfill
  In stochastic population models, the Poisson distribution can be
  used to model the number of births ($B_t$) that occur in a time
  interval. 
  \vfill
  \[
    B_t \sim \mathrm{Poisson}(N_t \times b)
  \]
\end{frame}


\begin{frame}[fragile]
  \frametitle{Poisson distributions}
\begin{knitrout}
\definecolor{shadecolor}{rgb}{0.969, 0.969, 0.969}\color{fgcolor}

{\centering \includegraphics[width=0.9\textwidth]{figure/pois-1} 

}


\end{knitrout}
\end{frame}



\begin{frame}
  \frametitle{Binomial distribution}
  The binomial distribution is also useful for data that are
  non-negative integers, but it has an upper bound. \\
  \vfill
  In population models, the upper bound is often population size, and
  we use the model to describe how many individuals die during some
  time period. \\
  \[
    D_t \sim \mathrm{Binomial}(N_t, d)
  \]
\end{frame}



\begin{frame}[fragile]
  \frametitle{Binomial distributions}
\begin{knitrout}
\definecolor{shadecolor}{rgb}{0.969, 0.969, 0.969}\color{fgcolor}

{\centering \includegraphics[width=0.9\textwidth]{figure/bin-1} 

}


\end{knitrout}
\end{frame}


\begin{frame}[fragile]
  \frametitle{Demographic stochasticity}
  Poisson-Binomial birth-death model. %\\
  \small
  \begin{gather*}
    N_{t+1} = N_t + B_t - D_t \\
    B_t \sim \mathrm{Poisson}(N_t \times b) \\
    D_t \sim \mathrm{Binomial}(N_t, d)
  \end{gather*}
\begin{knitrout}
\definecolor{shadecolor}{rgb}{0.969, 0.969, 0.969}\color{fgcolor}

{\centering \includegraphics[width=0.7\textwidth]{figure/pois-bin-1} 

}


\end{knitrout}
\end{frame}


\begin{frame}
  \frametitle{Summary}
  \Large
  Purely deterministic models are too rigid. \\
  \vfill
  Purely stochastic models don't describe population processes. \\
  \vfill
  The goal is to develop mechanistic models that represent 
  our biological understanding while allowing for uncertainty. \\
\end{frame}


%\begin{frame}
%  \frametitle{Assignment}
%  \Large
%   Read pages 19--22 in Conroy and Carroll
%\end{frame}





% quasi-equilibrium

%% \subsection{Birth-death process}


%% \begin{frame}
%%   \frametitle{Birth-death process}
%%   What does it mean for $r$ to be random?
%% \end{frame}


\end{document}


