%% UGA guidelines:
%% https://curriculumsystems.uga.edu/curriculum/courses/syllabus/

\documentclass[12pt]{article}

\usepackage[top=2.5cm,left=2.5cm,right=2.5cm,bottom=2cm]{geometry}
\usepackage[pdftex,colorlinks=true,linkcolor=blue,urlcolor=blue,pdfstartview={Fit}]{hyperref}
\usepackage{parskip}
\usepackage{setspace}
\usepackage{tocloft}
\usepackage{verbatim}



\begin{document}


%\begin{center}

{\centering

{\Large
  \bf \sc
  Applied Population Dynamics \\}
  WILD 5700/5700L, 7700/7700L, Fall 2023  \\
  Lecture: Mon, Wed 9:10--10:00 AM; Room 1-304 \\
  Lab: Wed 1:50--4:50; Room 1-304 \\


\normalsize

\vspace{0.5cm}

{\small
\begin{tabular}[h!]{lcccc}
& \textbf{Instructor}                 & \hspace{0.01cm} & \multicolumn{2}{c}{\textbf{Teaching Assistants}} \\
& Dr. Richard Chandler                & & Urvi Gupta & Joey Nolan \\
Office room: &  3-407                       & & 3-402  &     \\
% Phone: 762-499-4085                 & & --- & --- \\
Email: & rchandler@warnell.uga.edu    & & urvi.gupta@uga.edu &     \\
Office hours: & Mon 10:00--11:00 \& Tues 10:30--11:30 & & Mon 11:00-12:00  &  \\
%   Thurs 2:00--3:00 & & & \\
\end{tabular}
}

%\end{center}
}


%\vspace{0.1cm}

\normalsize


\vspace{-3mm}
\section*{\normalsize Course Description}
\vspace{-4mm}
This course will present the theory necessary for understanding
wildlife population dynamics, and it will explain how to use theory
and data to inform management and conservation efforts.

\vspace{-3mm}
\section*{\normalsize Course Objectives and Learning Outcomes}
\vspace{-4mm}
By the end of the course, students should know how to develop models
to forecast the impacts of environmental change and management actions
on wildlife populations. Students will learn how to design wildlife studies,
collect data, and estimate parameters such as abundance, survival, and
recruitment. 


\vspace{-3mm}
\section*{\normalsize Textbook}
\vspace{-4mm}
Conroy, M.J. and J.P. Carroll. 2009. Quantitative Conservation of
Vertebrates. Wiley-Blackwell. \href{http://preproxy.galib.uga.edu/login?url=http://onlinelibrary.wiley.com/book/10.1002/9781444303155}{Digital
  copies} are available for free through the UGA library. 

\vspace{-3mm}
\section*{\normalsize Grading}
\vspace{-4mm}
\begin{center}
%  \small
  \begin{tabular}[h!]{lrr}
    \hline
                        & Quantity & Grade percentage$^*$        \\
    \hline
    Quizzes             & 10       & 10\%                        \\
    Lab assignments     & 13       & 35\%                        \\
    Final paper$^{**}$  & 1        & 20\%                        \\
    Exams               & 3        & 30\%                        \\
    Class participation &          & 5\%                         \\
    \hline
  \end{tabular}                                                  \\
  \small
\hspace{0mm} $^*$Late assignments will be penalized 3 points/day \\ 
\hspace{0mm} $^{**}$Graduate students will analyze a real dataset and
summarize the results in their final paper.
\end{center}
\vspace{-6pt}
{%\small
The plus/minus grading system will be used: %, according to UGA policy,
%and assigned following this plus/minus grading scale:
A = $>$93-100, A- = $>$90-93, B+ = $>$87-90,                     \\ B = $>$83-87,
B- = $>$80-83, C+ = $>$77-80, C = $>$73-77, C-= $>$70-73, D = 60-70, F = $<$60. 
}



\vspace{-2mm}
\section*{\normalsize Academic Honesty}
\vspace{-4mm}

UGA Student Honor code: ``I will be academically honest in all of my
academic work and will not tolerate academic dishonesty of others.'' A
Culture of Honesty, the University's policy and procedures for
handling cases of suspected dishonesty, can be found at
\url{https://honesty.uga.edu/}. You are responsible for 
informing yourself about the university's standards before performing
any academic work. Lack of knowledge of the academic honesty policy is
not a reasonable explanation for a violation. Please ask if you have
questions related to course assignments and the academic honesty
policy. Any form of possible academic dishonesty will be reported to
the UGA Office of the Vice President for Instruction.

\clearpage


%\vspace{-12pt}
\section*{\normalsize Format and Attendance}
\vspace{-4mm}
Lectures and labs will be in-person. Attendance is optional, but it
will be impossible to succeed in the class if you don't participate in
lectures and labs. Quizzes will be taken during lectures. 

\vspace{-12pt}
\section*{\normalsize Cell Phones and Laptops}
\vspace{-4mm}
Cell phones are not allowed during class unless explicit permission is
granted. Laptop computers should be brought to class for quizzes and
exercises.


\vspace{-12pt}
\section*{\normalsize Academic Coaching}
\vspace{-4mm}

For assistance with time management, test and performance anxiety,
notetaking, motivation, text comprehension, test preparation, and
other barriers to success at UGA, visit the
\href{https://dae.uga.edu/services/academic-coaching/}{Office of Academic Enhancement}. 

\vspace{-12pt}
\section*{\normalsize Accommodations for Disabilities}
\vspace{-4mm}

If you require a disability-required accommodation, it is essential
that you register with the Disability Resource Center (\url{https://drc.uga.edu}; 706-542-8719)
and notify me of your eligibility for reasonable accommodations. We
can then plan how best to coordinate your accommodations. Please note
that accommodations cannot be provided retroactively.

% Accommodations for Disabilities (e.g., If you plan to request
% accommodations for a disability, please register with the Disability
% Resource Center. They can be reached by visiting Clark Howell Hall,
% calling 706-542-8719 (voice) or 706-542-8778 (TTY), or by visiting
% \url{http://drc.uga.edu}.)
% See \url{https://drc.uga.edu/content_page/sample-access-statements} for additional examples.


\vspace{-12pt}
\section*{\normalsize Mental Health and Wellness Resources}
\vspace{-4mm}

\begin{itemize}
  \setlength\itemsep{-6pt}
  \item If you or someone you know needs assistance, you are
    encouraged to contact \href{https://sco.uga.edu}{Student Care and
      Outreach} in the Division of Student Affairs (706-542-7774). They
    will help you navigate any difficult circumstances you may be facing
    by connecting you with the appropriate resources or services.
  \item UGA has several resources for a student seeking
    \href{https://www.uhs.uga.edu/bewelluga/bewelluga}{mental health}
    % mental health services (\url{https://www.uhs.uga.edu/bewelluga/bewelluga})
    or \href{https://www.uhs.uga.edu/info/emergencies}{crisis support}.% (\url{}).
  \item If you need help managing stress anxiety, relationships, etc.,
    please visit \href{https://www.uhs.uga.edu/bewelluga/bewelluga}{BeWellUGA} % (\url{})
    for a list of free workshops, classes, mentoring, and health
    coaching led by licensed clinicians and health educators in the
    University Health Center.
  \item Additional resources can be accessed through the UGA App.
\end{itemize}


\vspace{-12pt}
\section*{\normalsize Diversity and Inclusion}
\vspace{-4mm}
The phrase ‘diversity and inclusion’ encompasses welcoming and respect
for differences of culture, background and experience among
individuals and groups. Such differences include, but are not limited
to, differences of race, ethnicity, national origin, color, gender,
sexual orientation, gender identity, age, and abilities, as well as
political and religious affiliation and socioeconomic status. As UGA
grows more diverse, creating a welcoming climate of respect and
inclusiveness for all is more important than ever. 

\vspace{-12pt}
\section*{\normalsize Generative AI}
\vspace{-4mm}

% Students must disclose the use of large language models (LLMs) for
% completing assignments, including labs and the final paper. LLMs, like
% chatGPT and other generative AI chatbots, can only be used to improve
% text and code written by the student. Students must report the prompts
% provided to the LLM, along with the original text (or code) and
% the final text. Prompts and output can be disclosed by copying and
% pasting screenshots into assignment documents. 

Artificial intelligence (AI) tools, like chatGPT and other generative AI
chatbots have become popular for editing text and computer code. However,
they often produce incorrect output, and they can hinder learning if
used to generate answers and bypass critical thinking. All work
submitted in this course must be your own work. 
% Students must disclose the use of generative AI for
% completing assignments, including labs and the final paper.
Students can use generative AI to edit their own text and code,
but not to create answers by inputting assignment
questions into prompts. Students must disclose the prompts provided to 
AI tools, along with the original text (or code) and the final
text. Prompts and output can be disclosed by copying and pasting
screenshots into assignment documents. Failure to disclose this
information will be considered a violation of the UGA Student Honor
Code. 




%\newpage
\clearpage

\section*{\normalsize Tentative Course Outline}
%\vspace{-5mm}
%\large

\begin{center}
\begin{tabular}[c]{lll}
\hline \hline
{\bf Date} & {\bf Lecture}                      & {\bf Lab}                      \\
\hline
           \multicolumn{3}{c}{PART I -- Models}                                        \\
\hline
\hline
Aug 14     & Introduction                       & Excel and R basics; geometric growth \\
\hline
Aug 19     & BIDE model                         &                                      \\
Aug 21     & Exponential and geometric growth   & Logistic growth                      \\
\hline
Aug 26     & Logistic growth                    &                                      \\
Aug 28     & Harvest models                     & Harvest models                       \\
\hline
Sept 2     & Labor Day -- no lecture            &                                      \\
Sept 4     & Stochasticity                      & Extinction risk                      \\
\hline
Sept 9    & Extinction risk                    &                                      \\
Sept 11    & Exam I                             & Data collection                      \\
\hline
Sept 16    & Competition and predation          &                                      \\
Sept 18    & Competition and predation          & Competition and predation            \\
\hline
Sept 23    & Age structure                      &                                      \\
Sept 25    & Stage structure                    & Age/stage structure                  \\
\hline
Sept 30      & Source-sink dynamics               &                                      \\
Oct 1      & Metapopulations                    & Metapopulations                      \\
\hline
Oct 7     & Population viability analysis      &                                      \\
Oct 9     & Sensitivity analysis               & Population viability analysis        \\
\hline
           \multicolumn{3}{c}{PART II -- Estimation}                                   \\
\hline
Oct 14     & Study design                       &                                      \\
Oct 16     & Exam II                            & Data collection                      \\
\hline
Oct 21     & Occupancy models                   &                                      \\
Oct 23     & Occupancy models                   & Occupancy models                     \\
\hline
Oct 28     & Distance sampling                  & Due: First draft                     \\
Oct 30     & Distance sampling                  & Distance sampling                    \\
% \hline
% Nov 8    & Mark-recapture: closed-populations & Due: first draft of paper            \\
% Nov 11   & Mark-recapture: closed-populations & Mark-recapture I                     \\
\hline
Nov 4      & Mark-recapture: closed populations &                                      \\
Nov 6      & Mark-recapture: closed populations & Mark-recapture I                     \\
\hline
Nov 11     & Mark-recapture: open populations   & Due: peer review                     \\
Nov 13     & Mark-recapture: open populations   & Data collection                      \\
\hline
Nov 18     & Survival analysis                  &                                      \\
Nov 20     & Thanksgiving Break                 & Thanksgiving Break                   \\
\hline
Nov 25     & Case studies                       &                                      \\
Nov 27     & Case studies                       & Mark-recapture II                    \\
\hline
Dec 2      & Case studies                       & Due: Final paper                     \\
Dec 4      & Reading day                        & No lab                               \\
\hline
Dec 6     & Final exam (8:00-11:00)            &  Room 1-304                                 \\
\hline \hline
\end{tabular}
\end{center}

The course syllabus is a general plan for the course; deviations announced to the class by the instructor may be necessary.



\end{document}
