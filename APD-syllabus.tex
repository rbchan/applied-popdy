%% UGA guidelines:
%% https://curriculumsystems.uga.edu/curriculum/courses/syllabus/

\documentclass[12pt]{article}

\usepackage[top=2.5cm,left=2.5cm,right=2.5cm,bottom=2cm]{geometry}
\usepackage[pdftex,hidelinks,pdfstartview={Fit}]{hyperref}
\usepackage{parskip}
\usepackage{setspace}
\usepackage{tocloft}
\usepackage{verbatim}



\begin{document}


%\begin{center}

{\centering

{\Large
  \bf \sc
  Applied Population Dynamics \\}
  WILD 5700/5700L, 7700/7700L, Fall 2023  \\
  Lecture: Mon, Wed 9:10--10:00 AM; Room 1-304 \\
  Lab: Wed 1:50--4:50; Room 1-304 \\


\normalsize

\vspace{0.5cm}

{\small
\begin{tabular}[h!]{lcccc}
& \textbf{Instructor}                 & \hspace{0.01cm} & \multicolumn{2}{c}{\textbf{Teaching Assistants}} \\
& Dr. Richard Chandler                & & Urvi Gupta & Natalie Ramos \\
Office room: &  3-407                       & & 3-402  & 3-402    \\
% Phone: 762-499-4085                 & & --- & --- \\
Email: & rchandler@warnell.uga.edu    & & urvi.gupta@uga.edu & nar63365@uga.edu    \\
Office hours: & Mon 10:00--11:00 \& Thu 2:00--3:00 & & Mon 11:00-12:00  & Tues 10:00--11:00 \\
%   Thurs 2:00--3:00 & & & \\
\end{tabular}
}

%\end{center}
}


%\vspace{0.1cm}

\normalsize


\vspace{-3mm}
\section*{\normalsize Course Description}
\vspace{-4mm}
This course will present the theory necessary for understanding
wildlife population dynamics, and it will explain how to use theory
and data to inform management and conservation efforts.

\vspace{-3mm}
\section*{\normalsize Course Objectives and Learning Outcomes}
\vspace{-4mm}
By the end of the course, students should know how to develop models
to forecast the impacts of environmental change and management actions
on wildlife populations. Students will learn how to design wildlife studies,
collect data, and estimate parameters such as abundance, survival, and
recruitment. 


\vspace{-3mm}
\section*{\normalsize Textbook}
\vspace{-4mm}
Conroy, M.J. and J.P. Carroll. 2009. Quantitative Conservation of
Vertebrates. Wiley-Blackwell. Digital copies are available for free
through the UGA library: 
{\footnotesize \url{http://preproxy.galib.uga.edu/login?url=http://onlinelibrary.wiley.com/book/10.1002/9781444303155}}

\vspace{-3mm}
\section*{\normalsize Grading}
\vspace{-4mm}
\begin{center}
%  \small
  \begin{tabular}[h!]{lrr}
    \hline
                        & Quantity & Grade percentage$^*$        \\
    \hline
    Quizzes             & 10       & 10\%                        \\
    Lab assignments     & 13       & 35\%                        \\
    Final paper$^{**}$  & 1        & 20\%                        \\
    Exams               & 3        & 30\%                        \\
    Class participation &          & 5\%                         \\
    \hline
  \end{tabular}                                                  \\
  \small
\hspace{0mm} $^*$Late assignments will be penalized 3 points/day \\ 
\hspace{0mm} $^{**}$Graduate students will analyze a real dataset and
summarize the results in their final paper.
\end{center}
\vspace{-6pt}
{%\small
The plus/minus grading system will be used: %, according to UGA policy,
%and assigned following this plus/minus grading scale:
A = $>$93-100, A- = $>$90-93, B+ = $>$87-90,                     \\ B = $>$83-87,
B- = $>$80-83, C+ = $>$77-80, C = $>$73-77, C-= $>$70-73, D = 60-70, F = $<$60. 
}



\vspace{-2mm}
\section*{\normalsize Academic Honesty}
\vspace{-4mm}

UGA Student Honor code: ``I will be academically honest in all of my
academic work and will not tolerate academic dishonesty of others.'' A
Culture of Honesty, the University's policy and procedures for
handling cases of suspected dishonesty, can be found at
\url{https://honesty.uga.edu/}. You are responsible for 
informing yourself about the university's standards before performing
any academic work. Lack of knowledge of the academic honesty policy is
not a reasonable explanation for a violation. Please ask if you have
questions related to course assignments and the academic honesty
policy. Any form of possible academic dishonesty will be reported to
the UGA Office of the Vice President for Instruction.

\clearpage

\section*{\normalsize Generative AI}
\vspace{-4mm}

Students must disclose the use of large language models (LLMs) for
completing assignments, including labs and the final paper. LLMs, like
chatGPT and other generative AI chatbots, can only be used to improve
text and code written by the student. Students must report the prompts
provided to the LLM, along with the original text (or code) and
the final text. Prompts and output can be disclosed by copying and
pasting screenshots into assignment documents. 


\vspace{-12pt}
\section*{\normalsize Format and Attendance}
\vspace{-4mm}
Lectures and labs will be in-person. Attendance is optional, but it
will be impossible to succeed in the class if you don't participate in
lectures and labs. Quizzes will be taken during lectures. 

\vspace{-12pt}
\section*{\normalsize Cell Phones and Laptops}
\vspace{-4mm}
Cell phones are not allowed during class unless explicit permission is
granted. Laptop computers should be brought to class for quizzes and
exercises.


\vspace{-12pt}
\section*{\normalsize Academic Coaching}
\vspace{-4mm}

For assistance with time management, test and performance anxiety,
notetaking, motivation, text comprehension, test preparation, and
other barriers to success at UGA, visit the Office of Academic
Enhancement (\url{https://dae.uga.edu/services/academic-coaching/}). 

\vspace{-12pt}
\section*{\normalsize Accommodations for Disabilities}
\vspace{-4mm}

If you require a disability-required accommodation, it is essential
that you register with the Disability Resource Center (\url{https://drc.uga.edu}; 706-542-8719)
and notify me of your eligibility for reasonable accommodations. We
can then plan how best to coordinate your accommodations. Please note
that accommodations cannot be provided retroactively.

% Accommodations for Disabilities (e.g., If you plan to request
% accommodations for a disability, please register with the Disability
% Resource Center. They can be reached by visiting Clark Howell Hall,
% calling 706-542-8719 (voice) or 706-542-8778 (TTY), or by visiting
% \url{http://drc.uga.edu}.)
% See \url{https://drc.uga.edu/content_page/sample-access-statements} for additional examples.


\vspace{-12pt}
\section*{\normalsize Mental Health and Wellness Resources}
\vspace{-4mm}

\begin{itemize}
  \setlength\itemsep{-6pt}
  \item If you or someone you know needs assistance, you are
    encouraged to contact Student Care and Outreach in the Division of
    Student Affairs at 706-542-7774 or visit \url{https://sco.uga.edu}. They
    will help you navigate any difficult circumstances you may be facing
    by connecting you with the appropriate resources or services.
  \item UGA has several resources for a student seeking mental health
    services (\url{https://www.uhs.uga.edu/bewelluga/bewelluga}) or crisis
    support (\url{https://www.uhs.uga.edu/info/emergencies}).
  \item If you need help managing stress anxiety, relationships, etc.,
    please visit BeWellUGA (\url{https://www.uhs.uga.edu/bewelluga/bewelluga})
    for a list of FREE workshops, classes, mentoring, and health
    coaching led by licensed clinicians and health educators in the
    University Health Center.
  \item Additional resources can be accessed through the UGA App.
\end{itemize}


\vspace{-12pt}
\section*{\normalsize Diversity and Inclusion}
\vspace{-4mm}
The phrase ‘diversity and inclusion’ encompasses welcoming and respect
for differences of culture, background and experience among
individuals and groups. Such differences include, but are not limited
to, differences of race, ethnicity, national origin, color, gender,
sexual orientation, gender identity, age, and abilities, as well as
political and religious affiliation and socioeconomic status. As UGA
grows more diverse, creating a welcoming climate of respect and
inclusiveness for all is more important than ever. 



%\newpage
\clearpage

\section*{\normalsize Tentative Course Outline}
%\vspace{-5mm}
%\large

\begin{center}
\begin{tabular}[c]{lll}
\hline \hline
{\bf Date} & {\bf Lecture}                      & {\bf Lab}                      \\
\hline
           \multicolumn{3}{c}{PART I -- Models}                                        \\
\hline
%\note{Save BIDE model for second lecture. Take first few lectures
%  slow. Include Excel examples in lecture to prepare students for lab}
\hline
%Aug 24    & Introduction                       &                                      \\
Aug 16     & Introduction                       & Excel and R basics; geometric growth \\
%Aug 26    & BIDE model                         &                                      \\
\hline
Aug 21     & BIDE model                         &                                      \\
Aug 23     & Exponential and geometric growth   & Logistic growth                      \\
\hline
Aug 28     & Logistic growth                    &                                      \\
Aug 30     & Harvest models                     & Harvest models                       \\
\hline
Sept 4     & Labor Day -- no lecture            &                                      \\
Sept 6     & Stochasticity                      & Extinction risk                      \\
\hline
Sept 11    & Extinction risk                    &                                      \\
Sept 13    & Exam I                             & Data collection                      \\
\hline
Sept 18    & Competition and predation          &                                      \\
Sept 20    & Competition and predation          & Competition and predation            \\
\hline
Sept 25    & Age structure                      &                                      \\
Sept 27    & Stage structure                    & Age/stage structure                  \\
\hline
Oct 2      & Source-sink dynamics               &                                      \\
Oct 4      & Metapopulations                    & Metapopulations                      \\
\hline
Oct 9     & Population viability analysis      &                                      \\
Oct 11     & Sensitivity analysis               & Population viability analysis        \\
\hline
           \multicolumn{3}{c}{PART II -- Estimation}                                   \\
\hline
Oct 16     & Study design                       &                                      \\
Oct 18     & Exam II                            & Data collection                      \\
\hline
Oct 23     & Occupancy models                   &                                      \\
Oct 25     & Occupancy models                   & Occupancy models                     \\
\hline
Oct 30     & Distance sampling                  & Due: First draft                     \\
Nov 1      & Distance sampling                  & Distance sampling                    \\
% \hline
% Nov 8    & Mark-recapture: closed-populations & Due: first draft of paper            \\
% Nov 11   & Mark-recapture: closed-populations & Mark-recapture I                     \\
\hline
Nov 6      & Mark-recapture: closed populations &                                      \\
Nov 8      & Mark-recapture: closed populations & Mark-recapture I                     \\
\hline
Nov 13     & Mark-recapture: open populations   & Due: peer review                     \\
Nov 15     & Mark-recapture: open populations   & Data collection                      \\
\hline
Nov 20     & Survival analysis                  &                                      \\
Nov 22     & Thanksgiving Break                 & Thanksgiving Break                   \\
%           \multicolumn{3}{c}{Online Instruction Begins}                              \\
\hline
Nov 27     & Case studies                       &                                      \\
Nov 29     & Case studies                       & Mark-recapture II                    \\
\hline
Dec 4      & Case studies                       & Due: Final paper                     \\
Dec 6      & Reading day                        & No lab                               \\
\hline
Dec 13     & Final exam (8:00-11:00)            &                                      \\
\hline \hline
\end{tabular}
\end{center}

The course syllabus is a general plan for the course; deviations announced to the class by the instructor may be necessary.



\end{document}
