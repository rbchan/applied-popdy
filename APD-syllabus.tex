%% UGA guidelines:
%% https://curriculumsystems.uga.edu/curriculum/courses/syllabus/

\documentclass[12pt]{article}

\usepackage[top=2.5cm,left=2.5cm,right=2.5cm,bottom=2cm]{geometry}
\usepackage[pdftex,colorlinks=true,linkcolor=blue,urlcolor=blue,pdfstartview={Fit}]{hyperref}
\usepackage{parskip}
\usepackage{setspace}
\usepackage{tocloft}
\usepackage{verbatim}

\begin{document}

%\begin{center}

{\centering

{\Large
  \bf \sc
  Applied Population Dynamics \\}
  WILD 5700/5700L, 7700/7700L, Fall 2025  \\
  Lecture: Mon, Wed 9:10--10:00 AM; Room 1-304 \\
  Lab: Wed 1:50--4:50; Room 1-304 \\

\normalsize

\vspace{0.5cm}

{\small
\begin{tabular}[h!]{lcccc}
              & \textbf{Instructor}       & \hspace{0.01cm} & \multicolumn{2}{c}{\textbf{Teaching Assistants}} \\
              & Dr. Richard Chandler      &                 & Urvi Gupta         & Katherine Ara\'uz-Ponce     \\
Office room:  & 3-407                     &                 & 3-402              & 3-402                       \\
Email:        & rchandler@warnell.uga.edu &                 & urvi.gupta@uga.edu & arauzponce@uga.edu          \\
Office hours: & Mon 10:00--11:00          &                 & Thur 10:00--11:00  & Fri 11:30--12:30            \\
              & or by appointment         &                 &                    &                             \\
\end{tabular}
}

%\end{center}
}


%\vspace{0.1cm}

\normalsize


\vspace{-3mm}
\section*{\normalsize Course Description}
\vspace{-4mm}
% This course will present the theory necessary for understanding
% wildlife population dynamics, and it will explain how to use theory
% and data to inform management and conservation efforts.
Foundations of population dynamics from an applied
perspective. Exponential and logistic population growth, life tables,
species interactions, and stochastic models for structured
populations. Sampling design and analytical methods for estimating
abundance and demographic parameters. Application of population models
to harvest management and conversation of small
populations. Integrative student project required. 

\vspace{-3mm}
\section*{\normalsize Course Objectives and Learning Outcomes}
\vspace{-4mm}
% By the end of the course, students should know how to develop models
% to forecast the impacts of environmental change and management actions
% on wildlife populations. Students will learn how to design wildlife studies,
% collect data, and estimate parameters such as abundance, survival, and
% recruitment. 
This course will present the theory necessary for understanding
wildlife population dynamics, and it will explain how to use theory
and data to inform management and conservation efforts. By the end of 
the course, students will be able to: (1) Develop models to forecast
the impacts of environmental change and management actions on wildlife
populations; (2) Design field studies to collect data on wildlife
demographics; (3) Use statistical software to analyze data on wildlife
populations; (4) Review scientific literature to support modeling
decisions; (5) Present modeling results in a short scientific paper
and an oral presentation. 

% \begin{itemize}
%   \item Develop models to forecast the impacts of environmental change
%     and management actions on wildlife populations;
%   \item Design field studies to collect data on wildlife demographics;
%   \item Use statistical software to analyze data on wildlife populations;
%   \item Review scientific literature to support modeling decisions;
%   \item Present modeling results in a short scientific paper and an
%     oral presentation. 
% \end{itemize}

\vspace{-3mm}
\section*{\normalsize Prerequisites}
\vspace{-4mm}

\begin{tabular}{lp{0.8\linewidth}}
  Undergrads:& (MATH 1113 or MATH 1113E) and (FANR 3200 or FANR 3200W or
  ECOL 3500-3500L or ECOL 3505H-3505L) \\
  Grads:& FANR 6750-6750D or BIOS 7010 or BIOS 7010E \\
\end{tabular}

\vspace{-3mm}
\section*{\normalsize Textbook}
\vspace{-4mm}
Conroy, M.J. and J.P. Carroll. 2009. Quantitative Conservation of
Vertebrates. Wiley-Blackwell. \href{http://preproxy.galib.uga.edu/login?url=http://onlinelibrary.wiley.com/book/10.1002/9781444303155}{Digital
  copies} are available for free through the UGA library. 

\vspace{-3mm}
\section*{\normalsize Grading}
\vspace{-4mm}
\begin{center}
%  \small
  \begin{tabular}[h!]{lrr}
    \hline
                        & Quantity & Grade percentage$^*$        \\
    \hline
    Quizzes             & 10       & 10\%                        \\
    Lab assignments     & 13       & 35\%                        \\
    Final paper$^{**}$  & 1        & 20\%                        \\
    Exams               & 3        & 30\%                        \\
    Class participation &          & 5\%                         \\
    \hline
  \end{tabular}                                                  \\
  \small
\hspace{0mm} $^*$Late assignments will be penalized 3 points/day, up to a maximum of 50 points off. \\ 
\hspace{0mm} $^{**}$Graduate students will analyze a real dataset and
summarize the results in their final paper.
\end{center}
\vspace{-6pt}
{%\small
The plus/minus grading system will be used: %, according to UGA policy,
%and assigned following this plus/minus grading scale:
A = $>$93-100, A- = $>$90-93, B+ = $>$87-90,                     \\ B = $>$83-87,
B- = $>$80-83, C+ = $>$77-80, C = $>$73-77, C-= $>$70-73, D = 60-70, F = $<$60. 
}

\newpage

\vspace{-2mm}
\section*{\normalsize Academic Honesty}
\vspace{-4mm}

UGA Student Honor code: ``I will be academically honest in all of my
academic work and will not tolerate academic dishonesty of others.'' A
Culture of Honesty, the University's policy and procedures for
handling cases of suspected dishonesty, can be found at
\url{https://honesty.uga.edu/}. You are responsible for 
informing yourself about the university's standards before performing
any academic work. Lack of knowledge of the academic honesty policy is
not a reasonable explanation for a violation. Please ask if you have
questions related to course assignments and the academic honesty
policy. Any form of possible academic dishonesty will be reported to
the UGA Office of the Vice President for Instruction.

% \clearpage

%\vspace{-12pt}
\section*{\normalsize Format and Attendance}
\vspace{-4mm}
Lectures and labs will be in-person. Attendance is optional, but it
will be impossible to succeed in the class if you don't participate in
lectures and labs. Quizzes will be taken during lectures. 

%\vspace{-12pt}
\section*{\normalsize Cell Phones and Laptops}
\vspace{-4mm}
Cell phones are not allowed during class unless explicit permission is
granted. Laptop computers should be brought to class for quizzes and
exercises, but you should not use laptops for anything unrelated to
this course.


%\vspace{-12pt}
\section*{\normalsize UGA Well-Being Resources}
\vspace{-4mm}

UGA Well-being Resources promote student success by cultivating a
culture that supports a more active, healthy, and engaged student
community. 

Anyone needing assistance is encouraged to contact Student Care \&
Outreach (SCO) in the Division of Student Affairs at 706-542-8479 or
visit \url{sco.uga.edu}. Student Care \& Outreach helps students
navigate difficult circumstances by connecting them with the most
appropriate resources or services. They also administer the Embark@UGA
program which supports students experiencing, or who have experienced,
homelessness, foster care, or housing insecurity. 

UGA provides both clinical and non-clinical options to support student
well-being and mental health, any time, any place. Whether on campus,
or studying from home or abroad, UGA Well-being Resources are here to
help. 

% Well-being Resources: \url{well-being.uga.edu}

Student Care and Outreach: \url{sco.uga.edu}

University Health Center: \url{healthcenter.uga.edu}

Counseling and Psychiatric Services: \url{caps.uga.edu} or CAPS 24/7
crisis support at 706-542-2273 

Health Promotion/ Fontaine Center: \url{healthpromotion.uga.edu}

Accessibility and Testing: \url{accessibility.uga.edu}

Additional information, including free digital well-being resources,
can be accessed through the UGA app or by visiting
\url{https://well-being.uga.edu}. 

% \begin{itemize}
%   \setlength\itemsep{-6pt}
%   \item If you or someone you know needs assistance, you are
%     encouraged to contact \href{https://sco.uga.edu}{Student Care and
%       Outreach} in the Division of Student Affairs (706-542-7774). They
%     will help you navigate any difficult circumstances you may be facing
%     by connecting you with the appropriate resources or services.
%   \item UGA has several resources for a student seeking
%     \href{https://www.uhs.uga.edu/bewelluga/bewelluga}{mental health}
%     % mental health services (\url{https://www.uhs.uga.edu/bewelluga/bewelluga})
%     or \href{https://www.uhs.uga.edu/info/emergencies}{crisis support}.% (\url{}).
%   \item If you need help managing stress anxiety, relationships, etc.,
%     please visit \href{https://www.uhs.uga.edu/bewelluga/bewelluga}{BeWellUGA} % (\url{})
%     for a list of free workshops, classes, mentoring, and health
%     coaching led by licensed clinicians and health educators in the
%     University Health Center.
%   \item Additional resources can be accessed through the UGA App.
% \end{itemize}



%\vspace{-12pt}
\section*{\normalsize Academic Coaching}
\vspace{-4mm}

For assistance with time management, test and performance anxiety,
notetaking, motivation, text comprehension, test preparation, and
other barriers to success at UGA, visit the
\href{https://dae.uga.edu/services/academic-coaching/}{Office of Academic Enhancement}. 

%\vspace{-12pt}
\section*{\normalsize Accommodations for Disabilities}
\vspace{-4mm}

If you require a disability-required accommodation, it is essential
that you register with the Disability Resource Center
(\url{https://drc.uga.edu}; 706-542-8719) 
and notify me of your eligibility for reasonable accommodations. We
can then plan how best to coordinate your accommodations. Please note
that accommodations cannot be provided retroactively.



%\vspace{-12pt}
\section*{\normalsize Generative AI}
\vspace{-4mm}


Artificial intelligence (AI) tools, like chatGPT and other generative AI
chatbots have become popular for editing text and computer code. However,
they often produce incorrect output, and they can hinder learning if
used to generate answers and bypass critical thinking. All work
submitted in this course must be your own work, and it must make use
of the concepts and tools that we covered in class, rather than tools
(e.g., computer code) generated by AI. Students can use generative AI
to edit their own text and code, but not to create answers by
inputting assignment questions into prompts. Students must disclose
the prompts provided to  AI tools, along with the original text (or
code) and the final text. Prompts and output can be disclosed by
copying and pasting screenshots into assignment documents. Failure to
disclose this information will be considered a violation of the UGA
Student Honor Code.

More on AI:

\begin{itemize}
  \item \href{https://thebullshitmachines.com/}{Under the AI hood}
  \item \href{https://drive.google.com/file/d/1AN9GrksC4kLIE57fuS4zg8nPsyYtNEmr/view?usp=sharing}{AI
    teaching-learning circle}
\end{itemize}
\clearpage

\section*{\normalsize Tentative Course Outline}
%\vspace{-5mm}
%\large

\begin{center}
\begin{tabular}[c]{lll}
\hline \hline
{\bf Date} & {\bf Lecture}                      & {\bf Lab}                      \\
\hline
           \multicolumn{3}{c}{PART I -- Models}                                        \\
\hline
\hline
Aug 13     & Introduction                       & Excel and R basics; geometric growth \\
\hline
Aug 18     & BIDE model                         &                                      \\
Aug 20     & Exponential and geometric growth   & Logistic growth                      \\
\hline
Aug 25     & Logistic growth                    &                                      \\
Aug 27     & Harvest models                     & Harvest models                       \\
\hline
Sept 1     & Labor Day -- no lecture            &                                      \\
Sept 3     & Stochasticity                      & Extinction risk                      \\
\hline
Sept 8    & Extinction risk                    &                                      \\
Sept 10    & Exam I                             & Data collection                      \\
\hline
Sept 15    & Competition and predation          &                                      \\
Sept 17    & Competition and predation          & Competition and predation            \\
\hline
Sept 22    & Age structure                      &                                      \\
Sept 24    & Stage structure                    & Age/stage structure                  \\
\hline
Sept 29      & Source-sink dynamics               &                                      \\
Oct 1      & Metapopulations                    & Metapopulations                      \\
\hline
Oct 6     & Population viability analysis      &                                      \\
Oct 8     & Sensitivity analysis               & Population viability analysis        \\
\hline
           \multicolumn{3}{c}{PART II -- Estimation}                                   \\
\hline
Oct 13     & Study design                       &                                      \\
Oct 15     & Exam II                            & Data collection                      \\
\hline
Oct 20     & Occupancy models                   &                                      \\
Oct 22     & Occupancy models                   & Occupancy models                     \\
\hline
Oct 27     & Distance sampling                  & Due: First draft                     \\
Oct 29     & Distance sampling                  & Distance sampling                    \\
% \hline
% Nov 8    & Mark-recapture: closed-populations & Due: first draft of paper            \\
% Nov 11   & Mark-recapture: closed-populations & Mark-recapture I                     \\
\hline
Nov 3      & Mark-recapture: closed populations &                                      \\
Nov 5      & Mark-recapture: closed populations & Mark-recapture I                     \\
\hline
Nov 10     & Mark-recapture: open populations   & Due: peer review                     \\
Nov 12     & Mark-recapture: open populations   & Data collection                      \\
\hline
Nov 17     & Survival analysis                  &                                      \\
Nov 19     & Case studies                       & Mark-recapture II                   \\
\hline
Nov 24     & Case studies                       &                                      \\
Nov 26     & Thanksgiving Break                 & Thanksgiving Break                    \\
\hline
Dec 1      & Case studies                       & Due: Final paper                     \\
Dec 3      & Reading day                        & No lab                               \\
\hline
Dec 8     & Final exam (8:00-11:00)            &  Room 1-304                                 \\
\hline \hline
\end{tabular}
\end{center}

The course syllabus is a general plan for the course; deviations announced to the class by the instructor may be necessary.



\end{document}
