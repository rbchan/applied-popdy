%% UGA guidelines:
%% https://curriculumsystems.uga.edu/curriculum/courses/syllabus/

\documentclass[12pt]{article}

\usepackage[top=2.5cm,left=2.5cm,right=2.5cm,bottom=2cm]{geometry}
\usepackage[pdftex,hidelinks,pdfstartview={Fit}]{hyperref}
\usepackage{parskip}
\usepackage{setspace}
\usepackage{tocloft}
\usepackage{verbatim}



\begin{document}


%\begin{center}

{\centering

{\Large
 % \huge
  \bf \sc
  Applied Population Dynamics \\}
%\large
  WILD 5700/5700L, 7700/7700L  \\
  Lecture: Mon, Wed 9:05--9:55 AM; Room 4-517 \\
  Lab: Mon 1:25--4:25; Room 1-201 \\
%  Lab B: Fri 9:05--12:05; Room 1-201


\normalsize

\vspace{0.5cm}

\begin{tabular}[h!]{ccc}
\textbf{Instructor}                 & \hspace{1.5cm} & \textbf{Teaching Assistant} \\
Dr. Richard Chandler                & & Heather Levy \\
Office: 3-409-B                     & & Office: 3-402  \\
Phone: 706-542-5818                 & & --- \\
email: rchandler@warnell.uga.edu    & & email: heather.levy@uga.edu \\
Office hours: Wed 10:00-11:00 \& 1:30--2:30  & & Office hours: Tues 2:00-3:00 \\
\end{tabular}


%\end{center}
}


%\vspace{0.1cm}

\normalsize


\vspace{-2mm}
\section*{\normalsize Course Description}
\vspace{-4mm}
This course will present the theory necessary for understanding
wildlife population dynamics, and it will explain how to use theory
and data to inform management and conservation efforts.

\vspace{-2mm}
\section*{\normalsize Course Objectives and Learning Outcomes}
\vspace{-4mm}
By the end of the course, students should know how to develop models
to forecast the impacts of environmental change and management actions
on wildlife populations. Students will learn how to design wildlife studies,
collect data, and estimate parameters such as abundance,
survival, and recruitment.


\vspace{-2mm}
\section*{\normalsize Textbook}
\vspace{-4mm}
Conroy, M.J. and J.P. Carroll. 2009. Quantitative Conservation of
Vertebrates. Wiley-Blackwell.
% You {\bf do not} need to buy a hard copy.
Digital copies are available for free through the UGA library:
{\footnotesize \url{http://preproxy.galib.uga.edu/login?url=http://onlinelibrary.wiley.com/book/10.1002/9781444303155}}

\vspace{-2mm}
\section*{\normalsize Grading}
\vspace{-4mm}
\begin{center}
%  \small
  \begin{tabular}[h!]{lrr}
    \hline
                              & Quantity & Grade percentage      \\
    \hline
    Quizzes                   & 10       & 10\%                  \\
    Lab assignments$^*$       & 13       & 35\%                  \\
    Final paper$^{**}$         & 1        & 20\%                  \\
    Exams                     & 3        & 30\%                  \\
    Class participation       &          & 5\%                   \\
    \hline
  \end{tabular}                                                  \\
  \small
\hspace{0mm} $^*$Late assignments will be penalized 3 points/day \\ %%5\%/day. \\
\hspace{0mm} $^{**}$Graduate students will analyze a real dataset and
summarize the results in their final paper.
\end{center}



\vspace{-2mm}
\section*{\normalsize Academic Honesty}
\vspace{-4mm}
% As a University of Georgia student, you have agreed to abide by the
% University's academic honesty policy, ``A Culture of Honesty,'' and
% the Student Honor Code. All academic work must meet the standards
% described in ``A Culture of Honesty'' found at:
% \url{https://honesty.uga.edu/Academic-Honesty-Policy/}. Lack of
% knowledge of the academic honesty policy is not a reasonable
% explanation for a violation. Questions related to course assignments
% and the academic honesty policy should be directed to the instructor.
UGA Student Honor Code: ``I will be academically honest in all of my
academic work and will not tolerate academic dishonesty of others.'' A
Culture of Honesty, the University's policy and procedures for
handling cases of suspected dishonesty, can be found at
\url{www.uga.edu/ovpi}.   

\vspace{-2mm}
\section*{\normalsize Cell Phones and Laptops}
\vspace{-4mm}
Cell phones are not allowed during class unless explicit permission is
granted. Laptop computers should be brought to class for quizzes and
exercises.


%\newpage

\section*{Tentative Course Outline}
%\vspace{-5mm}
%\large

\begin{center}
\begin{tabular}[c]{lll}
\hline \hline
{\bf Date} & {\bf Lecture}                      & {\bf Lab (Mon)}               \\
\hline
           \multicolumn{3}{c}{PART I -- Models}                                     \\
\hline
%\note{Save BIDE model for second lecture. Take first few lectures
%  slow. Include Excel examples in lecture to prepare students for lab}
Jan 8      & Introduction / BIDE model          & No lab                            \\
\hline
Jan 13     & Exponential and geometric growth   & Excel basics and geometric growth \\
Jan 15     & Logistic growth                    &                                   \\
\hline
Jan 20     & No class                           & No lab
        \\
Jan 22     & Harvest models                     &                                   \\
\hline
Jan 27     & Stochasticity                      & Logistic growth and harvest       \\
Jan 29     & Extinction risk                    &                                   \\
\hline
Feb 3      & Competition and predation          & Extinction risk                   \\
Feb 5      & Competition and predation          &                                   \\
\hline
Feb 10     & Age structure                      & Competition and predation         \\
Feb 12     & Stage structure                    &                                   \\
\hline
Feb 17     & Source sink dynamics               & Age/stage structure               \\
Feb 19     & Exam I                             &                                   \\
\hline
Feb 24     & Ideal free/despotic distributions  & Data collection                   \\
Feb 26     & Metapopulations                    &                                   \\
\hline
Mar 2      & Metapopulations                    & Metapopulations                   \\
Mar 4      & Population viability analysis      &                                   \\
\hline
           \multicolumn{3}{c}{PART II -- Estimation}                                \\
\hline
Mar 9     & Spring break                       & Spring break                      \\
Mar 11     & Spring break                       &                                   \\
\hline
Mar 16     & Study design and statistics        & Occupancy models                  \\
Mar 18     & Occupancy models                   &                                   \\
\hline
Mar 23     & Occupancy models                   & Data collection                   \\
Mar 25     & Distance sampling                  &                                   \\
\hline
Mar 30      & Exam II                            & Distance sampling                 \\
Apr 1      & Distance sampling                  &                                   \\
\hline
Apr 6      & Mark-recapture: closed populations & Mark-recapture                    \\
Apr 8     & Mark-recapture: closed populations &                                   \\
\hline
Apr 13     & Survival analysis                  & Mark-recapture                    \\
Apr 15     & Mark-recapture: open populations   &                                   \\
\hline
Apr 20     & Mark-recapture: open populations   & Data collection                   \\
Apr 22     & Case studies                       &                                   \\
\hline
Apr 27     & Case studies                       & No lab                            \\
Apr 28     & Final paper due                    &                                   \\
\hline
May  1     & Final exam (8:00-11:00)            &                                   \\
\hline \hline
\end{tabular}
\end{center}

The course syllabus is a general plan for the course; deviations announced to the class by the instructor may be necessary.





\end{document}
