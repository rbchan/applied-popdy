%% UGA guidelines:
%% https://curriculumsystems.uga.edu/curriculum/courses/syllabus/

\documentclass[11pt]{article}

\usepackage[top=2.5cm,left=2.5cm,right=2.5cm,bottom=2cm]{geometry}
\usepackage[pdftex,hidelinks,pdfstartview={Fit}]{hyperref}
\usepackage{parskip}
\usepackage{setspace}
\usepackage{tocloft}
\usepackage{verbatim}



\begin{document}


\begin{center}

{%\LARGE 
  \huge \bf \sc 
  Applied Population Dynamics \\}
{\large 
  WILD 5700/5700L, 7700/7700L  \\
  Lecture: Mon, Wed 9:05--9:55 AM; Room 4-517 \\
  Lab A: Mon 1:25--4:25; Room 1-201
  Lab B: Fri 9:05--12:05; Room 1-201
}

\large

\vspace{0.5cm}

\begin{tabular}[h!]{ccc}
\textbf{Instructor}                 & \hspace{1.5cm} & \textbf{Teaching Assistant} \\
Dr. Richard Chandler                & & Brianna Williams \\
Office: 3-409-B                     & & Office: Wildlife Health building, room 112A  \\
Phone: 706-542-5818                 & & --- \\
email: rchandler@warnell.uga.edu    & & email: williams.brianna.m@gmail.com \\
Office hours: Thurs/Fri 2:00--3:00  & & Office hours: Wed 2:00--3:00 \\
\end{tabular}
\end{center}

%\vspace{0.1cm}

\normalsize


\vspace{-3mm}
\section*{Course Description}
\vspace{-5mm}
This course will present the theory necessary for understanding
wildlife population dynamics, and it will explain how to use theory
and data to inform management and conservation efforts. 

\vspace{-5mm}
\section*{Course Objectives and Learning Outcomes}
\vspace{-5mm}
By the end of the course, students should know how to develop models
to forecast the impacts of environmental change and management actions
on wildlife populations. Students will learn how to design studies,
collect data, and estimate parameters such as abundance,
survival, and recruitment.


\vspace{-5mm}
\section*{Textbook}
\vspace{-5mm}
Conroy, M.J. and J.P. Carroll. 2009. Quantitative Conservation of
Vertebrates. Wiley-Blackwell. You {\bf do not} need to buy a hard
copy. Digital copies are available for free through the UGA library:
{\footnotesize \url{http://preproxy.galib.uga.edu/login?url=http://onlinelibrary.wiley.com/book/10.1002/9781444303155}}

\vspace{-5mm}
\section*{Grading}
\vspace{-5mm}
\begin{center}
%  \small
  \begin{tabular}[h!]{lrr}
    \hline
                              & Quantity & Grade percentage  \\
    \hline
    Quizzes                   & 10       & 15\%              \\
    Lab assignments$^*$       & 13       & 30\%              \\
    Written assignment$^{**}$ & 1        & 20\%              \\
    Exams                     & 3        & 30\%              \\
    Class participation       &          & 5\%               \\
    \hline
  \end{tabular}                                              \\
  \small
\hspace{0mm} $^*$Late assignments will be penalized 5\%/day. \\
\hspace{0mm} $^{**}$Graduate students will analyze a real dataset and
summarize the results in their final paper. 
\end{center}



\vspace{-5mm}
\section*{Academic Honesty}
\vspace{-5mm}
As a University of Georgia student, you have agreed to abide by the
University's academic honesty policy, ``A Culture of Honesty,'' and
the Student Honor Code. All academic work must meet the standards
described in ``A Culture of Honesty'' found at:
\url{https://honesty.uga.edu/Academic-Honesty-Policy/}. Lack of
knowledge of the academic honesty policy is not a reasonable
explanation for a violation. Questions related to course assignments
and the academic honesty policy should be directed to the instructor.  


\vspace{-5mm}
\section*{Cell Phones and Laptops}
\vspace{-5mm}
Cell phones are not allowed during class unless explicit permission is
granted. Laptop computers should be brought to class for quizzes and
exercises.  


%\newpage

\section*{Tentative Course Outline}
\vspace{-5mm}
\large

\begin{center}
\begin{tabular}[c]{lll}
\hline \hline
{\bf Date} & {\bf Lecture}                      & {\bf Lab (Mon/Fri)}               \\
\hline
           \multicolumn{3}{c}{PART I -- Models}                                     \\
\hline
\note{Save BIDE model for second lecture. Take first few lectures
  slow. Include Excel examples in lecture to prepare students for lab}
Jan 9      & Introduction / BIDE model          & No lab                            \\ 
\hline
Jan 14     & Exponential and geometric growth   & Excel basics and geometric growth \\ 
Jan 16     & Logistic growth                    &                                   \\
\hline
Jan 21     & No class                           & No lab -- writing assignment      \\
Jan 23     & Harvest models                     &                                   \\
\hline
Jan 28     & Stochasticity                      & Logistic growth and harvest       \\
Jan 30     & Extinction risk                    &                                   \\
\hline
Feb 4      & Competition and predation          & Extinction risk                   \\
Feb 6      & Competition and predation          &                                   \\
\hline
Feb 11     & Age structure                      & Competition and predation         \\
Feb 13     & Stage structure                    &                                   \\
\hline 
Feb 18     & Source sink dynamics               & Age/stage structure               \\
Feb 20     & Exam I                             &                                   \\
\hline
Feb 25     & Ideal free/despotic distributions  & Data collection                   \\
Feb 27     & Metapopulations                    &                                   \\
\hline
Mar 4      & Metapopulations                    & Metapopulations                   \\
Mar 6      & Population viability analysis      &                                   \\
\hline
           \multicolumn{3}{c}{PART II -- Estimation}                                \\
\hline
Mar 11     & Spring break                       & Spring break                      \\
Mar 13     & Spring break                       &                                   \\
\hline
Mar 18     & Study design and statistics        & Occupancy models                  \\
Mar 20     & Occupancy models                   &                                   \\
\hline
Mar 25     & Occupancy models                   & Data collection                   \\
Mar 27     & Distance sampling                  &                                   \\
\hline
Apr 1      & Exam II                            & Distance sampling                 \\
Apr 3      & Distance sampling                  &                                   \\
\hline
Apr 8      & Mark-recapture: closed populations & Mark-recapture                    \\
Apr 10     & Mark-recapture: closed populations &                                   \\
\hline
Apr 15     & Survival analysis                  & Mark-recapture                    \\
Apr 17     & Mark-recapture: open populations   &                                   \\
\hline
Apr 22     & Mark-recapture: open populations   & Data collection                   \\
Apr 24     & Case studies                       &                                   \\
\hline
Apr 29     & Case studies                       & No lab -- Final paper due         \\
\hline
May  8     & Final exam (8:00-11:00)            &                                   \\ 
\hline \hline
\end{tabular}
\end{center}

The course syllabus is a general plan for the course; deviations announced to the class by the instructor may be necessary.





\end{document}
