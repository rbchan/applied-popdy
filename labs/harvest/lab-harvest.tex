\documentclass[12pt]{article}\usepackage[]{graphicx}\usepackage[]{color}
% maxwidth is the original width if it is less than linewidth
% otherwise use linewidth (to make sure the graphics do not exceed the margin)
\makeatletter
\def\maxwidth{ %
  \ifdim\Gin@nat@width>\linewidth
    \linewidth
  \else
    \Gin@nat@width
  \fi
}
\makeatother

\definecolor{fgcolor}{rgb}{0.345, 0.345, 0.345}
\newcommand{\hlnum}[1]{\textcolor[rgb]{0.686,0.059,0.569}{#1}}%
\newcommand{\hlstr}[1]{\textcolor[rgb]{0.192,0.494,0.8}{#1}}%
\newcommand{\hlcom}[1]{\textcolor[rgb]{0.678,0.584,0.686}{\textit{#1}}}%
\newcommand{\hlopt}[1]{\textcolor[rgb]{0,0,0}{#1}}%
\newcommand{\hlstd}[1]{\textcolor[rgb]{0.345,0.345,0.345}{#1}}%
\newcommand{\hlkwa}[1]{\textcolor[rgb]{0.161,0.373,0.58}{\textbf{#1}}}%
\newcommand{\hlkwb}[1]{\textcolor[rgb]{0.69,0.353,0.396}{#1}}%
\newcommand{\hlkwc}[1]{\textcolor[rgb]{0.333,0.667,0.333}{#1}}%
\newcommand{\hlkwd}[1]{\textcolor[rgb]{0.737,0.353,0.396}{\textbf{#1}}}%
\let\hlipl\hlkwb

\usepackage{framed}
\makeatletter
\newenvironment{kframe}{%
 \def\at@end@of@kframe{}%
 \ifinner\ifhmode%
  \def\at@end@of@kframe{\end{minipage}}%
  \begin{minipage}{\columnwidth}%
 \fi\fi%
 \def\FrameCommand##1{\hskip\@totalleftmargin \hskip-\fboxsep
 \colorbox{shadecolor}{##1}\hskip-\fboxsep
     % There is no \\@totalrightmargin, so:
     \hskip-\linewidth \hskip-\@totalleftmargin \hskip\columnwidth}%
 \MakeFramed {\advance\hsize-\width
   \@totalleftmargin\z@ \linewidth\hsize
   \@setminipage}}%
 {\par\unskip\endMakeFramed%
 \at@end@of@kframe}
\makeatother

\definecolor{shadecolor}{rgb}{.97, .97, .97}
\definecolor{messagecolor}{rgb}{0, 0, 0}
\definecolor{warningcolor}{rgb}{1, 0, 1}
\definecolor{errorcolor}{rgb}{1, 0, 0}
\newenvironment{knitrout}{}{} % an empty environment to be redefined in TeX

\usepackage{alltt}

%\usepackage[roman]{../pres1}
%\usepackage[sans]{../pres1}

\usepackage[hmargin=1in,vmargin=1in]{geometry}
\usepackage{parskip}
\usepackage{hyperref}
\hypersetup{pdfstartview=FitV,hidelinks}




\IfFileExists{upquote.sty}{\usepackage{upquote}}{}
\begin{document}

{
  \Large
  \centering
  Lab 3 Assignment --- Harvest Models \\
  Due before Tuesday \par
}

Answer each of the following questions and upload your completed Excel
file and R script to ELC. Be sure to show your calculations. Undergraduates
only need to do Exercise I in R.

\vspace{12pt}

{\bf Exercise I \\}
The Excel sheet shows (fake) data on Burmese python abundance in South Florida.
\begin{enumerate}
  \item What growth model best describes these data, geometric or
    logistic?  Hint: calculate $\lambda_t = N_t/N_{t-1}$ to assess if
    growth rates change over time.
  \item What is the growth rate ($r$)?
  \item If you determine that the per-capita birth rate ($b$) is 2.0,
    what must be the per-capita mortality rate ($d$)?
  \item What harvest rate ($h$) would result in a sustainable yield?
  \item Use the sustainable harvest rate ($h$) from part (4) to project the population
    forward from 2012 to 2024. You will need to compute the number of
    individuals removed ($H_t$) each year using the equation
    $H_t=N_{t}h$ (Note that $H_t$ can be greater than $N_t$ because
      harvest is assumed to occur at the end of the year, after the
      population has grown). Create a graph of python abundance from
    2005--2024.  
%  \item How is it possible that $H_t$ can be greater than $N_{t}$?
\end{enumerate}

\vspace{24pt}

Here's some R code to get you started on Exercise I:

\begin{knitrout}
\definecolor{shadecolor}{rgb}{0.969, 0.969, 0.969}\color{fgcolor}\begin{kframe}
\begin{alltt}
\hlstd{years1} \hlkwb{<-} \hlnum{2005}\hlopt{:}\hlnum{2024}                                  \hlcom{## All years}
\hlstd{nYears1} \hlkwb{<-} \hlkwd{length}\hlstd{(years1)}                            \hlcom{## Number of years}
\hlstd{nYearsWithData} \hlkwb{<-} \hlnum{8}                                  \hlcom{## nYears with data}
\hlstd{nYearsWithoutData} \hlkwb{<-} \hlstd{nYears1}\hlopt{-}\hlstd{nYearsWithData}          \hlcom{## nYears without data}
\hlstd{pythons} \hlkwb{<-} \hlkwd{c}\hlstd{(}\hlnum{10}\hlstd{,} \hlnum{25}\hlstd{,} \hlnum{63}\hlstd{,} \hlnum{156}\hlstd{,} \hlnum{391}\hlstd{,} \hlnum{977}\hlstd{,} \hlnum{2441}\hlstd{,} \hlnum{6104}\hlstd{,}  \hlcom{## Python counts}
             \hlkwd{rep}\hlstd{(}\hlnum{NA}\hlstd{, nYears1}\hlopt{-}\hlstd{nYearsWithData))}

\hlcom{## Question 1 hint: you can compute the first lambda like this:}
\hlstd{lambda} \hlkwb{<-} \hlkwd{rep}\hlstd{(}\hlnum{NA}\hlstd{, nYears1}\hlopt{-}\hlnum{1}\hlstd{)}
\hlstd{lambda[}\hlnum{1}\hlstd{]} \hlkwb{<-} \hlstd{pythons[}\hlnum{2}\hlstd{]}\hlopt{/}\hlstd{pythons[}\hlnum{1}\hlstd{]}
\hlcom{## You could use a 'for loop' to compute lambda in each year}

\hlcom{## For question 5, you will want a loop like this:}
\hlkwa{for}\hlstd{(t} \hlkwa{in} \hlstd{(nYearsWithData}\hlopt{+}\hlnum{1}\hlstd{)}\hlopt{:}\hlstd{nYears) \{}
   \hlcom{## PUT CODE HERE}
\hlstd{\}}
\end{alltt}
\end{kframe}
\end{knitrout}



\clearpage



{\bf Exercise II \\}
Imagine a population of northern bobwhite ({\it Colinus virginianus})
that is experiencing logistic growth with $r_\mathrm{max} = 0.32$, $K = 2000$,
and an initial population size of 100 individuals.
\begin{enumerate}
  \item Project the population for 40 years, and plot abundance over
    time. Add axis labels as always.
  \item Compute the number of individuals that {\it could be} sustainably
    harvested each year. Plot abundance on the x-axis and sustainable
    harvest on the y-axis.
  \item At what value of abundance ($N$) would maximum sustainable yield (MSY) occur?
  \item What is the value of MSY in this case? %What is the harvest rate ($h$) at MSY?
  \item Using the same values of $r_\mathrm{max}$, $K$, and $N_0$, project the population
    forward again, but include harvest ($H_t$). Choose values of $H_t$ that
    allow for the greatest number of years at MSY. Hint: You can let
    harvest be zero in some years.
\end{enumerate}

%\vspace{12pt}

\newpage


{\bf Exercise III \\}
Suppose that annual survival of sitka deer ({\it Odocoileus hemionus
  sitkensis}) decreases as abundance increases according to the
equation: $S = \beta_0 - \beta_1 \times N$.

\begin{enumerate}
  \item Compute survival probability for each value of $N$ provided in
    the spreadsheet with $\beta_0=0.95$ and $\beta_1=0.003$. Create a
    graph with survival probability on the y-axis and abundance on the
    x-axis.
  \item A manager is trying to decide how many deer to harvest, and is
    considering removing anywhere from 10 to 150 individuals from a
    population of 200. Use the equation above to determine how many
    deer will remain one year after harvest for each of the harvest
    options. To accomplish this:
  \begin{itemize}
    \item Compute how many individuals will be alive immediately after
      harvest
    \item Compute survival probability for these remaining individuals
      using the survival equation
    \item Compute how many will be alive at the end of the year.
  \end{itemize}
  \item Create a graph with final abundance ($N$) on the y-axis and
    harvest ($H$) of the x-axis. %Why does this graph peak at
    %intermediate levels of harvest?
  \item Determine how many deer will be alive if no harvest
    occurs. Are there any levels of harvest that can result in a
    larger population than the no harvest scenario? If so, how can
    this be?
  \item If the manager's objective is to maximize harvest, while
    maintaining a herd size greater than it would be without harvest,
    how many deer should be taken?
\end{enumerate}


\newpage

{\bf R tips \\}


Here's an example of a logistic growth model.
\begin{knitrout}\small
\definecolor{shadecolor}{rgb}{0.969, 0.969, 0.969}\color{fgcolor}\begin{kframe}
\begin{alltt}
\hlstd{rmax} \hlkwb{<-} \hlnum{0.1}              \hlcom{## max growth rate}
\hlstd{K} \hlkwb{<-} \hlnum{200}                 \hlcom{## carrying capacity}
\hlstd{years} \hlkwb{<-} \hlnum{2001}\hlopt{:}\hlnum{2050}       \hlcom{## years}
\hlstd{nYears} \hlkwb{<-} \hlkwd{length}\hlstd{(years)}
\hlstd{N1} \hlkwb{<-} \hlkwd{rep}\hlstd{(}\hlnum{NA}\hlstd{, nYears)}
\hlstd{N1[}\hlnum{1}\hlstd{]} \hlkwb{<-} \hlnum{100}             \hlcom{## abundance in first year}
\hlkwa{for}\hlstd{(t} \hlkwa{in} \hlnum{2}\hlopt{:}\hlstd{nYears) \{}
    \hlstd{N1[t]} \hlkwb{<-} \hlstd{N1[t}\hlopt{-}\hlnum{1}\hlstd{]} \hlopt{+} \hlstd{N1[t}\hlopt{-}\hlnum{1}\hlstd{]}\hlopt{*}\hlstd{rmax}\hlopt{*}\hlstd{(}\hlnum{1} \hlopt{-} \hlstd{N1[t}\hlopt{-}\hlnum{1}\hlstd{]}\hlopt{/}\hlstd{K)}
\hlstd{\}}
\end{alltt}
\end{kframe}
\end{knitrout}



%\newpage


Here's an example of a logistic growth model with harvest
\begin{knitrout}\small
\definecolor{shadecolor}{rgb}{0.969, 0.969, 0.969}\color{fgcolor}\begin{kframe}
\begin{alltt}
\hlstd{h} \hlkwb{<-} \hlnum{0.02}                \hlcom{## harvest rate}
\hlstd{N2} \hlkwb{<-} \hlkwd{rep}\hlstd{(}\hlnum{NA}\hlstd{, nYears)}
\hlstd{N2[}\hlnum{1}\hlstd{]} \hlkwb{<-} \hlnum{100}             \hlcom{## abundance in first year}
\hlstd{H} \hlkwb{<-} \hlkwd{rep}\hlstd{(}\hlnum{NA}\hlstd{, nYears}\hlopt{-}\hlnum{1}\hlstd{)}
\hlkwa{for}\hlstd{(t} \hlkwa{in} \hlnum{2}\hlopt{:}\hlstd{nYears) \{}
    \hlstd{H[t}\hlopt{-}\hlnum{1}\hlstd{]} \hlkwb{<-} \hlstd{N2[t}\hlopt{-}\hlnum{1}\hlstd{]}\hlopt{*}\hlstd{h}
    \hlstd{N2[t]} \hlkwb{<-} \hlstd{N2[t}\hlopt{-}\hlnum{1}\hlstd{]} \hlopt{+} \hlstd{N2[t}\hlopt{-}\hlnum{1}\hlstd{]}\hlopt{*}\hlstd{rmax}\hlopt{*}\hlstd{(}\hlnum{1} \hlopt{-} \hlstd{N2[t}\hlopt{-}\hlnum{1}\hlstd{]}\hlopt{/}\hlstd{K)} \hlopt{-} \hlstd{H[t}\hlopt{-}\hlnum{1}\hlstd{]}
\hlstd{\}}
\hlkwd{plot}\hlstd{(years, N1,} \hlkwc{type}\hlstd{=}\hlstr{"l"}\hlstd{,} \hlkwc{xlab}\hlstd{=}\hlstr{"Year"}\hlstd{,} \hlkwc{ylab}\hlstd{=}\hlstr{"Abundance"}\hlstd{)}
\hlkwd{lines}\hlstd{(years, N2,} \hlkwc{col}\hlstd{=}\hlstr{"blue"}\hlstd{,} \hlkwc{lty}\hlstd{=}\hlnum{2}\hlstd{)}
\hlkwd{legend}\hlstd{(}\hlnum{2000}\hlstd{,} \hlnum{200}\hlstd{,} \hlkwd{c}\hlstd{(}\hlstr{"Logistic growth"}\hlstd{,} \hlstr{"Logistic growth with harvest"}\hlstd{),}
       \hlkwc{col}\hlstd{=}\hlkwd{c}\hlstd{(}\hlstr{"black"}\hlstd{,} \hlstr{"blue"}\hlstd{),} \hlkwc{lty}\hlstd{=}\hlkwd{c}\hlstd{(}\hlnum{1}\hlstd{,}\hlnum{2}\hlstd{))}
\end{alltt}
\end{kframe}

{\centering \includegraphics[width=0.5\textwidth]{figure/logistic-harvest-1} 

}


\end{knitrout}


\end{document}

