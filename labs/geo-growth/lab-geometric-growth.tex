\documentclass[12pt]{article}\usepackage[]{graphicx}\usepackage[]{xcolor}
% maxwidth is the original width if it is less than linewidth
% otherwise use linewidth (to make sure the graphics do not exceed the margin)
\makeatletter
\def\maxwidth{ %
  \ifdim\Gin@nat@width>\linewidth
    \linewidth
  \else
    \Gin@nat@width
  \fi
}
\makeatother

\definecolor{fgcolor}{rgb}{0.345, 0.345, 0.345}
\newcommand{\hlnum}[1]{\textcolor[rgb]{0.686,0.059,0.569}{#1}}%
\newcommand{\hlsng}[1]{\textcolor[rgb]{0.192,0.494,0.8}{#1}}%
\newcommand{\hlcom}[1]{\textcolor[rgb]{0.678,0.584,0.686}{\textit{#1}}}%
\newcommand{\hlopt}[1]{\textcolor[rgb]{0,0,0}{#1}}%
\newcommand{\hldef}[1]{\textcolor[rgb]{0.345,0.345,0.345}{#1}}%
\newcommand{\hlkwa}[1]{\textcolor[rgb]{0.161,0.373,0.58}{\textbf{#1}}}%
\newcommand{\hlkwb}[1]{\textcolor[rgb]{0.69,0.353,0.396}{#1}}%
\newcommand{\hlkwc}[1]{\textcolor[rgb]{0.333,0.667,0.333}{#1}}%
\newcommand{\hlkwd}[1]{\textcolor[rgb]{0.737,0.353,0.396}{\textbf{#1}}}%
\let\hlipl\hlkwb

\usepackage{framed}
\makeatletter
\newenvironment{kframe}{%
 \def\at@end@of@kframe{}%
 \ifinner\ifhmode%
  \def\at@end@of@kframe{\end{minipage}}%
  \begin{minipage}{\columnwidth}%
 \fi\fi%
 \def\FrameCommand##1{\hskip\@totalleftmargin \hskip-\fboxsep
 \colorbox{shadecolor}{##1}\hskip-\fboxsep
     % There is no \\@totalrightmargin, so:
     \hskip-\linewidth \hskip-\@totalleftmargin \hskip\columnwidth}%
 \MakeFramed {\advance\hsize-\width
   \@totalleftmargin\z@ \linewidth\hsize
   \@setminipage}}%
 {\par\unskip\endMakeFramed%
 \at@end@of@kframe}
\makeatother

\definecolor{shadecolor}{rgb}{.97, .97, .97}
\definecolor{messagecolor}{rgb}{0, 0, 0}
\definecolor{warningcolor}{rgb}{1, 0, 1}
\definecolor{errorcolor}{rgb}{1, 0, 0}
\newenvironment{knitrout}{}{} % an empty environment to be redefined in TeX

\usepackage{alltt}

\usepackage[hmargin=1in,vmargin=1in]{geometry}
\usepackage{parskip}
\usepackage{hyperref}
\hypersetup{pdfstartview=FitV,hidelinks}




\IfFileExists{upquote.sty}{\usepackage{upquote}}{}
\begin{document}


{
  \Large
  \centering
  Lab 2 assignment --- Geometric growth models \\
  Due by 5:00pm on Friday \par
}

Answer the following questions by completing the provided Excel 
template and by replicating your work in R. Undergraduates should do
Exercise 1 in R, but not Exercise 2. Upload your Excel file and a
single R script to ELC. Name the files with your last
name followed by your first name. \\ 

\vspace{12pt}

{\bf Exercise 1 \\}
A study was conducted in which a population of 100 Mexican wolves
({\it Canis lupus baileyi}) was monitored intensively for 1 year. In
that time, 10 pups were born, and 20 of the original 100 individuals 
died. There was no immigration or emigration.  

\begin{enumerate}
  \item What are the values of $B$, $D$, $b$, $d$, $r$, and lambda
    ($\lambda$)?
  \item Assuming geometric growth, what will population size be in
    each of the subsequent 10 years? Create a graph of the results,
    including important chart elements such as axis titles. Remember,
    the geometric growth model is: $N_{t+1} = N_t + N_t r$.
\end{enumerate}

\vspace{24pt}

{
  \centering
  \includegraphics[width=0.7\textwidth]{figs/Coronadopack2} \\
}


\clearpage

% {\bf Exercise 2 \\}

% Abundance data on harbor seals ({\it Phoca vitulina}) in the Wadden
% Sea are provided from 1997--2009.

% \begin{enumerate}
%   \item Assuming logistic growth, find values of $r_{max}$ and $K$
%     that provide a good fit to the data. Use trial error to find these
%     values by comparing the actual number of seals to the number of
%     seals predicted by your model. You don't have to be too
%     precise, but try to get the predicted values within $\pm 2$ of the
%     actual data.
%   \item Use your model to predict how many seals will be in the
%     population in the year 2020.
%   \item Create a graph with year on the x-axis and abundance on the
%     y-axis. Show the data and the model predictions as separate
%     lines. 
% \end{enumerate}

% \vspace{24pt}

% {
%   \centering
%   \includegraphics[width=0.8\textwidth]{figs/pacific-harbor-seal} \\
% }



% \clearpage

{\bf Exercise 2 \\}
Many people assume that the human population acts differently than
most wildlife populations and is increasing at an exponential rate, or
close to it. The UN estimated that the human population was
approximately 7.02 billion in 2010, and 7.11 billion in 2011.


\begin{enumerate}
  \item Given these two years of data and the geometric growth equation:
    $N_{t+1} = N_t + N_t r$, calculate $r$ for the human population. 
  \item Use the geometric growth equation with the value of $r$ that
    you calculated above to predict the human population from 2012 to
    2025. 
  \item Graph your predictions of the human population along with the
    actual human population over time.
  \item Calculate the growth rate lambda ($\lambda_t = N_t / N_{t-1}$)
    for each year and for both the predicted human population and the
    actual human population data. Graph both values of lambda from
    2011 to 2025.
  \item Based on these results, did the human population exhibit
    geometric growth? Write 2-3 sentences to defend your answer. 
\end{enumerate}

\vspace{24pt}

More data about the human population: \\
\url{https://www.worldometers.info/world-population/} \\
\url{https://ourworldindata.org/world-population-growth}.

%% {\bf Exercise 2b \\}
%% UN data on human population size from 1999 to 2012 are provided on
%% sheet ``Exercise 2b''.


%% \begin{enumerate}
%%   \item Calculate the population growth rate (``lambda'':
%%     $\lambda_t = N_{t}/N_{t-1}$) for each year starting in 2000.
%%   \item Using a starting values of 6 billion people in 1999, pick
%%     values of $r$ that make a geometric growth curve
%%     closely approximate the actual data. Use trial and error to find a
%%     good value of $r$. 
%%   \item The logistic growth model is: 
%%     $N_{t+1} = N_t + N_t r_{\rm max}(1 - N_t/K)$ where $r_{\rm max}$
%%     is the maximum growth rate and $K$ is the carrying capacity.
%%     Using a starting values of 6 billion people in 1999, pick
%%     values of $r_{max}$ and $K$ that make a logistic growth curve
%%     closely approximate the actual data. Again, use trial and error to
%%     find good parameter values. Start by choosing a carrying capacity
%%     ($K$) that seems sensible for the human population.  
%%   \item Use the values you calculated above to make four graphs of:
%%     \begin{itemize}
%%       \item Actual abundance ($N$) over time
%%       \item Lambda ($\lambda_t$) over time
%%       \item Predicted geometric growth over time
%%       \item Predicted logistic growth over time.
%%     \end{itemize}
%%   \item Is recent human population growth more similar to geometric or
%%     logistic growth? Be explicit about how the observed trends of
%%     abundance and $\lambda_t$ indicate how the human population is
%%     growing. Write your answer (3--4 sentences) in the Excel sheet.
%% \end{enumerate}



\newpage

{\bf R tips \\}


Here's an example of a geometric growth model.
\begin{knitrout}
\definecolor{shadecolor}{rgb}{0.969, 0.969, 0.969}\color{fgcolor}\begin{kframe}
\begin{alltt}
\hldef{r} \hlkwb{<-} \hlnum{0.1}                      \hlcom{# growth rate}
\hldef{years} \hlkwb{<-} \hlnum{2001}\hlopt{:}\hlnum{2050}            \hlcom{# years}
\hldef{nYears} \hlkwb{<-} \hlkwd{length}\hldef{(years)}       \hlcom{# number of years}
\hldef{N} \hlkwb{<-} \hlkwd{rep}\hldef{(}\hlnum{NA}\hldef{, nYears)}          \hlcom{# empty vector}
\hldef{N[}\hlnum{1}\hldef{]} \hlkwb{<-} \hlnum{100}                   \hlcom{# abundance in first year}
\hlkwa{for}\hldef{(t} \hlkwa{in} \hlnum{2}\hlopt{:}\hldef{nYears) \{}
    \hldef{N[t]} \hlkwb{<-} \hldef{N[t}\hlopt{-}\hlnum{1}\hldef{]} \hlopt{+} \hldef{N[t}\hlopt{-}\hlnum{1}\hldef{]}\hlopt{*}\hldef{r}
\hldef{\}}
\hlkwd{plot}\hldef{(years, N,} \hlkwc{type}\hldef{=}\hlsng{"l"}\hldef{,} \hlkwc{xlab}\hldef{=}\hlsng{"Year"}\hldef{,} \hlkwc{ylab}\hldef{=}\hlsng{"Abundance"}\hldef{,}
     \hlkwc{main}\hldef{=}\hlsng{"Geometric growth"}\hldef{)}
\end{alltt}
\end{kframe}

{\centering \includegraphics[width=0.8\textwidth]{figure/geo-1} 

}


\end{knitrout}



%% Logistic growth.
%% <<logistic,out.width="0.6\\textwidth",fig.align="center",size='small'>>=
%% rmax <- 0.1              ## max growth rate
%% K <- 200                 ## carrying capacity
%% years <- 2001:2050       ## years
%% nYears <- length(years)
%% N <- rep(NA, nYears)
%% N[1] <- 100             ## abundance in first year
%% ## for loop
%% for(t in 2:nYears) {
%%     N[t] <- N[t-1] + N[t-1]*rmax*(1 - N[t-1]/K)
%% }
%% plot(years, N, type="l", xlab="Year", ylab="Abundance",
%%      main="Logistic growth")
%% @ 



\end{document}

